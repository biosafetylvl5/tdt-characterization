\documentclass{ssiBio}
\usepackage{comment}
\title{TdT Characterization Experiment} % CHANGE THIS
\author{Written by \textbf{Michael Uttmark}}
\date{\textbf{Written:} 2/16/19 \quad\textbf{Performed:} TBD\quad\textbf{Printed:} \today{}}

\begin{document}
\maketitle

\setVar{TotalInitialVolume}{15uL}
\setVar{BSStartConc}{100uM}
\setVar{BSStartMolWeight}{6133g/mol}
\setVar{dTTPConc}{10mM}

\setVar{TdTBuffVol}{1.5uL}
\setVar{CoCl2Vol}{1.5uL}
\setVar{TdTVol}{2uL}
\setVar{DNASolVol}{5uL}
\setVar{NucSolVol}{5uL}
\setVar{EDTAVol}{5uL}

\setVar{PANDNAConc}{2ng/uL}
\setVar{TdTSmudgeFactor}{3}
\setVar{SampleSmudgeFactor}{100}


\setVar{Time1}{30s} % total time for tdt reaction.
\setVar{Time2}{45s}
\setVar{Time3}{1min}
\setVar{Time4}{2min}
\setVar{Time5}{3min}
\setVar{Time6}{5min}
\setVar{Time7}{7min}
\setVar{Time8}{10min}
\setVar{Time9}{15min}
\setVar{Time-10}{20min}
\setVar{Time-11}{30min}

\setVar{nuConA}{2} % Nucleotide:DNA ratio in moles
\setVar{nuConB}{3}
\setVar{nuConC}{5}
\setVar{nuConD}{7}
\setVar{nuConE}{10}
\setVar{nuConF}{15}
\setVar{nuConG}{20}
\setVar{nuConH}{40}

\setCalcVar{RawDNAMolPerSample}{((((PANDNAConc*TdTSmudgeFactor)/(BSStartMolWeight))*(TotalInitialVolume+EDTAVol))) to pmol}
\setCalcVar{RawDNAuLPerSample}{(RawDNAMolPerSample*(BSStartConc)^-1) to uL}
\setCalcVar{DNADilute}{RawDNAuLPerSample*SampleSmudgeFactor to uL}
\setCalcVar{H2ODilute}{DNASolVol*SampleSmudgeFactor-DNADilute to uL}

\setVar{dTTPDiluteTwoConc}{1mM}
\setVar{dTTPDiluteConc}{.1mM}
\setVar{dTTPSmudgeFactor}{2}

\setVar{dTTPDilInter}{RawDNAMolPerSample*((dTTPDiluteConc)^-1)*12*dTTPSmudgeFactor}
\setVar{dTTPDilTwoInter}{RawDNAMolPerSample*((dTTPDiluteTwoConc)^-1)*12*dTTPSmudgeFactor}

\setVar{DNAContSolVol}{20uL}
\setVar{DNAStockConc}{100uM}
\setCalcVar{TotalDNAContVol}{2000uL*1}%{DNAContSolVol*96*1.05}
\setVar{LongDNA}{90-mer (AGTATTCCTGTCCCCGCTCGGTATCGTGATGCTCGTGCTTATGCGAACGTGCAGATGACGTTTGGCGCGTATTTAGATGCGCTAGACGTT)}
\setVar{ShortDNA}{BS-Comp-8 (AACGCACA)}
\setVar{MedDNA}{15Mer (ATGGACATGGACTAC)}
\setVar{LongWeightDNA}{27786g/mol}
\setVar{MedWeightDNA}{4601g/mol}
\setVar{ShortWeightDNA}{2387g/mol}
\setVar{ShortWDNA}{0.25}
\setVar{MedWDNA}{0.5}
\setVar{LongWDNA}{1}

%\setVar{PANDNAConc}{2ng/uL}
\setCalcVar{LongVolDNA}{(LongWeightDNA*DNAStockConc*((TotalDNAContVol)^-1)*DNAContSolVol^-1*((30uL))*(PANDNAConc*LongWDNA)^-1)^-1 to uL}
\setCalcVar{MedVolDNA}{(MedWeightDNA*DNAStockConc*((TotalDNAContVol)^-1)*DNAContSolVol^-1*((30uL))*(PANDNAConc*MedWDNA)^-1)^-1 to uL}
\setCalcVar{ShortVolDNA}{(ShortWeightDNA*DNAStockConc*((TotalDNAContVol)^-1)*DNAContSolVol^-1*((30uL))*(PANDNAConc*ShortWDNA)^-1)^-1 to uL}
\setCalcVar{DNAContWaterVol}{TotalDNAContVol-(LongVolDNA+ShortVolDNA+MedVolDNA)}

\section{Procedure Purpose}
Charecterize the rate of \tdt{} extension of ssDNA with varying nucleotide:DNA ratios.
\section{Overview}
An array of samples will be tested with these conditions (dTTP:DNA ratio (moles) by time of reaction):

\begin{center}
    \begin{tabular}{|c|c|c|c|c|c|c|c|c|c|c|c|c|}
        \hline
        & \getVar{Time1} & \getVar{Time2} & \getVar{Time3} & \getVar{Time4} & \getVar{Time5} & \getVar{Time6} & \getVar{Time7} & \getVar{Time8} & \getVar{Time9} & \getVar{Time-10} & \getVar{Time-11} & Controls\\\hline
        \getVar{nuConA}&&&&&&&&&&&&\\\hline
        \getVar{nuConB}&&&&&&&&&&&&\\\hline
        \getVar{nuConC}&&&&&&&&&&&&\\\hline
        \getVar{nuConD}&&&&&&&&&&&&\\\hline
        \getVar{nuConE}&&&&&&&&&&&&\\\hline
        \getVar{nuConF}&&&&&&&&&&&&\\\hline
        \getVar{nuConG}&&&&&&&&&&&&\\\hline
        \getVar{nuConH}&&&&&&&&&&&&H12\\\hline
    \end{tabular}
\end{center}

Samples are addressed by their 96Well plate location (eg. H12, as marked)

They will be stopped with EDTA and sent to PAN. The most promising samples will be run on a gel for better resolution.

For PAN analysis, we will be sending in \tdt{} extended samples with a concentration of \calc{PANDNAConc to ng/uL}.
\section{Safety Information}
\begin{safety}
\begin{enumerate}
\tdtSafety{}
\SYBRGOLD{}
\item{Working in a communal lab space is dangerous. Do not assume your fellow workers cleaned up sufficiently.}
\end{enumerate}
\end{safety}

\section{Dilutions}
\begin{enumerate}
    \item{Dilute \getVar{DNADilute} of BSStart (\getVar{BSStartConc}) in \getVar{H2ODilute}. Votex Well.}
    \item{Dilute \getVar{dTTPConc} dTTP Stock to 100\uL{} of \getVar{dTTPDiluteConc} (making at least \calc{(nuConA+nuConB+nuConC+nuConD+nuConE+nuConF)*(dTTPDilInter) to uL})}
    \item{Dilute \getVar{dTTPConc} dTTP Stock to 100\uL{} of \getVar{dTTPDiluteTwoConc} (making at least \calc{(nuConG+nuConH)*dTTPDilTwoInter to uL})}
    \item{Dilute \getVar{LongVolDNA} of \getVar{LongDNA}, \getVar{MedVolDNA} of \getVar{MedDNA} and \getVar{ShortVolDNA} of \getVar{ShortDNA} (both from \getVar{DNAStockConc} stock) into \getVar{DNAContWaterVol} Nuclease-Free water. Vortex well before use, this is the DNA Control Solution.}
\end{enumerate}

\section{Procedure}% CHANGE THIS
\subsection{Prepare Plate}
In to each well, pipette \getVar{TdTBuffVol} of 10X \tdt{} buffer and \getVar{CoCl2Vol} of \ce{CoCl2}. Alternatively, if possible (reagents allow), use a multichannel pipette to pipette in \calc{CoCl2Vol+TdTBuffVol to uL} of a 1:1 \ce{CoCl2}:10X TdT Buffer mixture into each cell.

For each row, create the following dilutions of dTTP from dilutes prepared before:

\begin{center}
    \begin{tabular}{|c|c|c|c|}
        \hline
        Row&dTTP Dilute (\getVar{dTTPDiluteConc})& dTTP Dilute (\getVar{dTTPDiluteTwoConc})&Water\\\hline
        A&\calc{nuConA*dTTPDilInter to uL}&&\calc{(12*dTTPSmudgeFactor*NucSolVol)-nuConA*dTTPDilInter to uL}\\\hline
        B&\calc{nuConB*dTTPDilInter to uL}&&\calc{(12*dTTPSmudgeFactor*NucSolVol)-nuConB*dTTPDilInter to uL}\\\hline
        C&\calc{nuConC*dTTPDilInter to uL}&&\calc{(12*dTTPSmudgeFactor*NucSolVol)-nuConC*dTTPDilInter to uL}\\\hline
        D&\calc{nuConD*dTTPDilInter to uL}&&\calc{(12*dTTPSmudgeFactor*NucSolVol)-nuConD*dTTPDilInter to uL}\\\hline
        E&\calc{nuConE*dTTPDilInter to uL}&&\calc{(12*dTTPSmudgeFactor*NucSolVol)-nuConE*dTTPDilInter to uL}\\\hline
        F&\calc{nuConF*dTTPDilInter to uL}&&\calc{(12*dTTPSmudgeFactor*NucSolVol)-nuConF*dTTPDilInter to uL}\\\hline
        G&&\calc{nuConG*dTTPDilTwoInter to uL}&\calc{(12*dTTPSmudgeFactor*NucSolVol)-nuConG*dTTPDilTwoInter to uL}\\\hline
        H&&\calc{nuConH*dTTPDilTwoInter to uL}&\calc{(12*dTTPSmudgeFactor*NucSolVol)-nuConH*dTTPDilTwoInter to uL}\\\hline
    \end{tabular}
\end{center}

For each column \textbf{except for 12} pipette \getVar{TdTVol} of \tdt{} into each well.

For each well in each row, pipette \getVar{NucSolVol} of the corresponding dilute.

\subsection{Begin Expiriment}

For each time, begin by pipetting in \getVar{DNASolVol} of the DNA dilute prepared earlier. Wait until the allotted time for that column has elapsed and cease the reaction with \getVar{EDTAVol} of EDTA.

\subsection{Controls}

No \tdt{} should be added to the controls. H12 must be left empty.

\subsection{Sending to PAN}

Pipette 10uL out of each well and store in another well plate for a future gel run. Pipette in \getVar{DNAContSolVol} of DNA Control Solution into each well and submit to PAN. (This gives us roughly 60ng of DNA per lane to run on a gel later)

\section{Stop Procedure}
Store the DNA 10 uM stocks in the -20 freezer immediately after use. Ensure that the SYBR Gold has been returned to the -20 freezer, and clean up the work area. Save all samples for possible future use.
Return polyamines to the cold room. Do not dispose of serial dilutions unless through a licensed chemical professional.

\begin{comment}
\subsection{Analysis}

\begin{figure}[ht]
\centering
\includegraphics[width=6in]{./gels/Gel_1.png}
\label{}
\caption{Gel 1}
\end{figure}

\bibliographystyle{ieeetr}
\bibliography{main}
\end{comment}
\end{document}
